

\section*{Kurzdarstellung}
\label{sec:kurzdarstellung}

\subsection{Was ist zu tun}
Kurze Zusammenfassung der Arbeit, höchstens halbe Seite.
Nenne die Zielsetzung, die Problemstellung und die Forschungsfragen. Wenn deiner Abschlussarbeit bestimmte Hypothesen zugrunde liegen, erwähne diese auch.

\subsubsection*{Beispiele}
Abstract

Unternehmen verfolgen zunehmend das Ziel, Marketingkampagnen zur Vermarktung ihrer Produkte einzusetzen. 
Um online einen wachsenden Umsatz zu generieren, greifen sie insbesondere auf Social-Media-Kanäle zurück. 
Daher ist es notwendig, zu verstehen, wie solche Marketingkampagnen konzipiert sind und wie sie funktionieren.
Das Ziel in der vorliegenden Arbeit ist es, zu klären, durch welche Kriterien eine Marketingkampagne in Social Media erfolgreich verläuft. Dazu wird die folgende Forschungsfrage gestellt: Wie kann eine erfolgreiche Marketingkampagne für Onlinefotodruckunternehmen in Social Media geplant werden?

Um die Forschungsfrage zu beantworten, wurde eine quantitative Studie zu aktuellen Druckgeschäftsanzeigen und deren Wirkung durchgeführt. Insbesondere wurde in der Studie auf Anzeigen aus den Social-Media-Kanälen Twitter, Facebook und Instagram Bezug genommen. Es wurde untersucht, welche Kriterien bei einer Anzeige erfüllt sein müssen, damit diese bei den Nutzenden erfolgreich ist. In der quantitativen Studie waren den Teilnehmenden geschlossene Fragen mit Antwortmöglichkeiten auf einer Skala von 1 bis 10 gestellt worden, die im Anschluss ausgewertet wurden. Dabei wurden die Teilnehmenden nach ihrem Alter in drei Gruppen unterteilt: 15–29-Jährige, 30–45-Jährige und über 45-Jährige.

Die Antworten auf die Fragen zeigen, dass die Altersgruppe der 30–45-Jährigen und die der über 45-Jährigen im Durchschnitt am häufigsten auf die Anzeigen von Onlinefotodruckunternehmen reagieren. Diese Anzeigen sind in erster Linie auf Twitter und Facebook erfolgreich, weil diese Plattformen von Personen dieser Altersklassen am häufigsten genutzt werden. Jüngere Menschen hingegen, die vorwiegend Instagram verwenden, reagieren seltener auf die Anzeigen von Onlinefotodruckunternehmen. Eine Social-Media-Kampagne bietet sich für Onlinefotodruckunternehmen also insbesondere auf Twitter und Facebook mit der Fokussierung auf die Altersgruppe ab 30 Jahren an.

Weiterführende Forschung im Bereich des Marketings für den Onlinefotodruck könnte auf Anzeigenwerbung von Suchmaschinen ausgerichtet sein.

\subsection{Kurzdarstellung}
Das Ziel in der vorliegenden Arbeit ist es, zu klären, durch welche\dots


\section{Anleitungen}
In this chapter, we're actually using some code!

\begin{lstlisting}[language=Python,caption={This is an example of inline listing},captionpos=b]
x = 1
if x == 1:
    # indented four spaces
    print("x is 1.")

\end{lstlisting}

You can also include listings from a file directly:

\lstinputlisting[language=Python,caption={This is an example of included listing},captionpos=b]{listings/example.py}

\thispagestyle{empty}
You may have read about similar things in \cite{Goodliffe2007}.
You can also write footnotes.\footnote{Footnotes will be positioned automatically.}




\subsection{And an even more important subsection}

It is possible to reference glossary entries as \gls{library} as an example.
