
\thispagestyle{empty}
\section*{Kurzdarstellung}
\label{sec:kurzdarstellung}

\subsection{Was ist zu tun}
Kurze Zusammenfassung der Arbeit, höchstens halbe Seite.
Nenne die Zielsetzung, die Problemstellung und die Forschungsfragen. Wenn deiner Abschlussarbeit bestimmte Hypothesen zugrunde liegen, erwähne diese auch.
% https://www.scribbr.de/aufbau-und-gliederung/abstract-schreiben/


\subsection{Kurzdarstellung}
Das Ziel in der vorliegenden Arbeit ist es, zu klären, durch welche\dots


\section{Anleitungen}
In this chapter, we're actually using some code!

\begin{lstlisting}[language=Python,caption={This is an example of inline listing},captionpos=b]
x = 1
if x == 1:
    # indented four spaces
    print("x is 1.")

\end{lstlisting}

You can also include listings from a file directly:

\lstinputlisting[language=Python,caption={This is an example of included listing},captionpos=b]{listings/example.py}

You may have read about similar things in \cite{Goodliffe2007}.
You can also write footnotes.\footnote{Footnotes will be positioned automatically.}
You may have read about similar things in \cite{Goodliffe2007}.
You can also write footnotes.\footnote{Footnotes will be positioned automatically.}




\subsection{And an even more important subsection}

It is possible to reference glossary entries as \gls{library} as an example.
