\chapter{Einleitung}\label{ch:Einleitung}

Das Metaverse, ein umfassender virtueller Raum, der durch die Kombination physischer und virtueller Realität entsteht, hat in den letzten Jahren erhebliche Aufmerksamkeit auf sich gezogen. Es wird als die nächste große Entwicklung des Internets angesehen, die eine völlig neue Dimension der Interaktion und des Erlebens ermöglicht. Mit Technologien wie Virtual Reality (VR), Augmented Reality (AR) und Künstlicher Intelligenz (KI) schafft das Metaverse immersive Umgebungen, in denen Nutzer arbeiten, spielen und soziale Kontakte pflegen können (Lee et al., 2021). Facebooks Umbenennung in Meta im Jahr 2021 und die damit verbundene Investition in die Entwicklung des Metaverse verdeutlichen die Bedeutung und das Potenzial dieser Technologie (Meta, 2021).

Jedoch bringt diese neue digitale Welt auch zahlreiche Herausforderungen und Risiken mit sich. Eine besonders bedrohliche Form der Cyberkriminalität im Metaverse ist das Social Engineering. Social Engineering bezieht sich auf die Manipulation von Menschen, um vertrauliche Informationen zu erlangen oder sie zu Handlungen zu bewegen, die ihre Sicherheit gefährden (Hadnagy, 2010). Im Metaverse, wo die Grenzen zwischen Realität und Virtualität verschwimmen, können Angreifer besonders raffinierte Methoden einsetzen, um ihre Ziele zu erreichen (Smith, 2023).


\section{Problemstellung}
Die Problemstellung dieser Arbeit ergibt sich aus der zunehmenden Verbreitung und Nutzung des Metaverse, welche neue Angriffsvektoren für Social Engineering eröffnet. Angreifer können die immersive Natur des Metaverse ausnutzen, um Vertrauen zu gewinnen und Nutzer zu täuschen. Die Anonymität und die komplexen sozialen Interaktionen im Metaverse erleichtern es den Angreifern, sich als vertrauenswürdige Personen oder Organisationen auszugeben. Dies führt zu erheblichen Risiken für die Privatsphäre und Sicherheit der Nutzer (Wilson, 2022).

Ein Beispiel für die Gefahr von Social Engineering im Metaverse ist die Nutzung von Deep Fakes, um gefälschte, aber äußerst realistische Avatare oder Videos zu erstellen. Diese können verwendet werden, um Nutzer zu täuschen und sie dazu zu bringen, sensible Informationen preiszugeben oder schädliche Aktionen durchzuführen (Chesney und Citron, 2019). Darüber hinaus können durch Gamification-Elemente im Metaverse Nutzer manipuliert und zu bestimmten Verhaltensweisen verleitet werden, ohne dass sie sich der Manipulation bewusst sind (Bec, 2022).

\section{Zielsetzung der Arbeit}

Ziel dieser Arbeit ist es, die Gefahren des Social Engineerings im Kontext des Metaverse umfassend zu analysieren und mögliche Schutzmaßnahmen zu erörtern. Dabei sollen folgende Forschungsfragen im Fokus stehen:


\begin{itemize}
\item Welche spezifischen Social Engineering-Techniken werden im Metaverse eingesetzt?
\item Welche Sicherheitslücken und Schwachstellen machen das Metaverse anfällig für Social Engineering-Angriffe?
\item Welche Maßnahmen können ergriffen werden, um die Nutzer und Systeme im Metaverse besser zu schützen?
\end{itemize}

Um diese Fragen zu beantworten, wird eine Kombination aus Literaturrecherche und Fallstudien verwendet. Die Arbeit soll einen fundierten Überblick über die aktuellen Bedrohungen und möglichen Lösungsansätze geben und dabei sowohl technische als auch organisatorische und psychologische Aspekte berücksichtigen.

Ein weiterer Schwerpunkt der Arbeit liegt auf der Untersuchung der Auswirkungen von Social Engineering-Angriffen im Metaverse. Dies umfasst die persönlichen, sozialen und wirtschaftlichen Folgen für die Betroffenen sowie die gesellschaftlichen Implikationen (Naughton, 2021). Darüber hinaus sollen Empfehlungen für die Entwicklung und Implementierung effektiver Schutzmaßnahmen gegeben werden, um die Sicherheit im Metaverse zu erhöhen und das Bewusstsein für die Risiken zu schärfen.