\chapter{Schutzmechanismen und Abwehrstrategien}\label{ch:SchutzmechanismenundAbwehrstrategien}

\section{Technische Sicherheitsmaßnahmen}
\subsection*{copy}
• Verschlüsselung: Einsatz von Ende-zu-Ende-Verschlüsselung für Datenübertragungen innerhalb des Metaverse, um die Datensicherheit und Privatsphäre zu gewährleisten.\\
• Authentifizierung und Zugriffskontrolle: Verstärkung der Sicherheitsprotokolle durch Mehrfaktor-Authentifizierung und regelmäßige Überprüfung der Zugriffsrechte, um sicherzustellen, dass nur autorisierte Nutzer Zugang zu sensiblen Bereichen oder Informationen haben.\\
• Anomalieerkennung und Überwachung: Implementierung von Systemen zur Erkennung ungewöhnlicher Aktivitäten oder Verhaltensweisen, die auf einen Social Engineering-Angriff hindeuten könnten.\\

• Zwei-Faktor-Authentifizierung (2FA): Eine zusätzliche Sicherheitsebene für den Zugang zu virtuellen Umgebungen, die über das einfache Passwort hinausgeht.\\
• Ende-zu-Ende-Verschlüsselung: Sicherstellung, dass Kommunikation zwischen den Nutzern nicht von Dritten eingesehen werden kann.\\
• Regelmäßige Sicherheitsaudits: Überprüfung und Aktualisierung der Sicherheitseinstellungen und -protokolle, um Schwachstellen zu identifizieren und zu beheben.\\


\subsection*{eigene}


\section{Aufklärung und Bewusstseinsbildung}
TODO

Brillen können gehackt werden Biometrische Daten ausgelesen werden wie mimiken etc 

\section{Deep Fakes}
TODO

Avatar ist ein Gegenstand und kann lauschen
Technische Sicherheitsmaßnahmen
