\chapter{Social Engineering}\label{ch:SocialEngineering}

\section{Geschichte des Social Engineering}

\subsection*{copy}

\paragraph*{1}
Die Geschichte des Social Engineerings ist eng mit der Entwicklung menschlicher Kommunikation und Interaktion verbunden. Social Engineering bezeichnet in diesem Kontext die Kunst der Manipulation von Menschen, um sie dazu zu bringen, vertrauliche Informationen preiszugeben oder bestimmte Handlungen auszuführen. Diese Praktik kann für verschiedene Zwecke eingesetzt werden, von Spionage und Betrug bis hin zur Sicherheitsanalyse. Im Folgenden skizziere ich einige Schlüsselmomente und Entwicklungen in der Geschichte des Social Engineerings, die für deine Bachelorarbeit relevant sein könnten:

Frühe Geschichte und Kriegsführung: Bereits in antiken Geschichten und Kriegen spielte Social Engineering eine Rolle, etwa wenn Spione Informationen sammelten oder wenn durch List und Täuschung Kriege gewonnen wurden. Ein berühmtes Beispiel aus der griechischen Mythologie ist das Trojanische Pferd, das als Strategie betrachtet werden kann, den Feind durch Täuschung zu besiegen.

19. und frühes 20. Jahrhundert: Mit der Industrialisierung und der zunehmenden Komplexität der Gesellschaft nahmen auch Betrug und Täuschung zu. Berühmte Betrüger wie Victor Lustig, der angeblich den Eiffelturm "verkaufte", nutzten Social Engineering-Techniken, um ihre Betrügereien durchzuführen.

Zweiter Weltkrieg: Im Zweiten Weltkrieg wurde Social Engineering von verschiedenen Geheimdiensten eingesetzt, um Spionage zu betreiben und feindliche Operationen zu stören. Die Operation Fortitude, bei der die Alliierten die Deutschen über den tatsächlichen Ort der D-Day-Invasion täuschten, ist ein Beispiel für erfolgreiches Social Engineering.

Das Zeitalter der Informationstechnologie: Mit dem Aufkommen von Computern und dem Internet hat sich Social Engineering weiterentwickelt. Phishing-Angriffe, bei denen Betrüger versuchen, über gefälschte E-Mails sensible Informationen zu erlangen, sind ein gängiges Beispiel. Kevin Mitnick, ein berühmter Hacker, nutzte in den 1980er und 1990er Jahren Social Engineering, um in Netzwerke einzudringen, indem er sich als Mitarbeiter ausgab und Passwörter oder andere kritische Informationen erbeutete.

21. Jahrhundert: Heute ist Social Engineering ein kritisches Element der Cybersecurity. Unternehmen und Organisationen müssen sich gegen eine Vielzahl von Social Engineering-Angriffen wappnen, darunter Spear-Phishing, Pretexting und Baiting. Die zunehmende Vernetzung und Digitalisierung bieten Angreifern immer neue Angriffsflächen.

Ausbildung und Gegenmaßnahmen: Die Erkenntnis, dass der menschliche Faktor oft das schwächste Glied in der Sicherheitskette ist, hat zur Entwicklung von Schulungsprogrammen und Gegenmaßnahmen geführt. Unternehmen investieren in die Ausbildung ihrer Mitarbeiter, um sie über die Risiken des Social Engineerings aufzuklären und ihnen beizubringen, wie sie Angriffe erkennen und vermeiden können.

In deiner Bachelorarbeit könntest du darauf eingehen, wie Social Engineering-Techniken sich im Laufe der Zeit verändert haben und welche Methoden heute am effektivsten sind. Außerdem könnte die Untersuchung der psychologischen Aspekte, die Social Engineering so wirkungsvoll machen, sowie der Gegenstrategien, die Organisationen entwickeln, um sich zu schützen, interessante Themenbereiche sein.

Für eine fundierte Bachelorarbeit zum Thema Social Engineering sind akademische und zuverlässige Quellen essenziell. Hier sind einige empfohlene Ressourcen, die als Ausgangspunkt für deine Recherche dienen können. Bitte beachte, dass du für die aktuellsten Informationen und spezifische Fallstudien auch auf Artikel in Fachzeitschriften und Konferenzberichte zugreifen solltest.

\paragraph*{Bücher und Monografien:}
Hadnagy, Christopher. "Social Engineering: The Art of Human Hacking." Wiley, 2010. Dieses Buch bietet einen umfassenden Überblick über die verschiedenen Techniken des Social Engineerings und wie sie angewendet werden.
Mitnick, Kevin D., und William L. Simon. "Die Kunst der Täuschung: Risikofaktor Mensch." Mitp-Verlags GmbH u. Co. Kg, 2003. Kevin Mitnick, einer der berühmtesten Hacker, teilt seine Erfahrungen und gibt Einblick in die Methoden des Social Engineerings.
Wissenschaftliche Artikel:

Workman, Michael. "A Test of Interventions for Security Threats from Social Engineering." Information u. Management, 45(8), 2008, Seiten 507-512. Dieser Artikel untersucht, wie Bildung und Bewusstsein die Anfälligkeit für Social Engineering-Angriffe reduzieren können.
Gragg, Derrick. "A Multi-Level Defense Against Social Engineering." SANS Institute InfoSec Reading Room, 2003. Ein technischer Bericht, der Strategien zur Abwehr von Social Engineering bietet.
Offizielle Dokumentationen und Richtlinien:

National Institute of Standards and Technology (NIST). "Guide to Malware Incident Prevention and Handling for Desktops and Laptops." NIST Special Publication 800-83, Revision 2. NIST Publikationen bieten Richtlinien und Best Practices für die IT-Sicherheit, einschließlich des Schutzes gegen Social Engineering.
Bundesamt für Sicherheit in der Informationstechnik (BSI). "IT-Grundschutz-Kompendium." Das BSI bietet umfassende Richtlinien zur IT-Sicherheit, einschließlich Empfehlungen zum Umgang mit Social Engineering.

Für eine vertiefende Auseinandersetzung mit der Geschichte des Social Engineerings und dessen vielschichtigen Aspekten gibt es einige Schlüsselwerke, die du in deiner Forschung heranziehen kannst. Diese Bücher bieten sowohl historische Einblicke als auch praktische Beispiele und theoretische Rahmenbedingungen, um das Phänomen des Social Engineerings umfassend zu verstehen:

"Social Engineering: The Art of Human Hacking" von Christopher Hadnagy
Veröffentlicht: 2010
Verlag: Wiley
ISBN: 978-0470639535
Inhalt: Dieses Buch bietet eine umfassende Einführung in die Techniken des Social Engineerings, illustriert durch echte Beispiele und Fallstudien. Hadnagy diskutiert sowohl die psychologischen Grundlagen als auch die Anwendung von Social Engineering in verschiedenen Kontexten.


The Art of Deception: Controlling the Human Element of Security von Kevin Mitnick und William L. Simon
Veröffentlicht: 2002
Verlag: Wiley
ISBN: 978-0764542800
Inhalt: Kevin Mitnick, einer der bekanntesten Hacker und ehemaligen Social Engineers, teilt seine Erfahrungen und beschreibt detailliert, wie Social Engineering-Angriffe durchgeführt werden. Das Buch hebt die Bedeutung der menschlichen Psychologie hervor und bietet Einblicke, wie man sich gegen solche Angriffe schützen kann.


"Ghost in the Wires: My Adventures as the Worlds Most Wanted Hacker" von Kevin Mitnick
Veröffentlicht: 2011
Verlag: Little, Brown and Company
ISBN: 978-0316037709
Inhalt: Dieses Buch ist eine Autobiografie von Kevin Mitnick, die seine Karriere als Hacker nachzeichnet und dabei viele Aspekte des Social Engineerings beleuchtet, einschließlich detaillierter Beschreibungen von seinen berühmtesten Hacks und den dabei angewandten Social Engineering Techniken.

"Phishing for Phools: The Economics of Manipulation and Deception" von George A. Akerlof und Robert J. Shiller
Veröffentlicht: 2015
Verlag: Princeton University Press
ISBN: 978-0691168319
Inhalt: Obwohl dieses Buch sich primär auf wirtschaftliche Manipulation konzentriert, bietet es wertvolle Einblicke in die Mechanismen und Strategien, die auch im Social Engineering eine Rolle spielen. Es erklärt, wie Menschen zu Entscheidungen verleitet werden, die nicht in ihrem besten Interesse sind.

\paragraph*{Online-Ressourcen}
SANS Institute Reading Room. https://www.sans.org/reading-room/ Der Reading Room des SANS Institute enthält eine Fülle von Forschungsartikeln und Berichten zu verschiedenen Aspekten der Cybersicherheit, einschließlich Social Engineering.
Cybersecurity und Infrastructure Security Agency (CISA). https://www.cisa.gov/ CISA bietet Ressourcen und Alerts zu aktuellen Bedrohungen und Anfälligkeiten, auch bezüglich Social Engineering.
Bitte beachte, dass der Zugang zu einigen akademischen Artikeln und Büchern eingeschränkt sein kann und möglicherweise über Bibliotheken oder akademische Datenbanken wie JSTOR, Google Scholar oder die Datenbank deiner Universität zugänglich ist. Es ist auch empfehlenswert, die Zitationen in diesen Quellen zu prüfen, um weitere relevante Literatur zu finden.

SANS Institute InfoSec Reading Room
URL: \url{https://www.sans.org/reading-room/}
Inhalt: Der Reading Room bietet eine Vielzahl von Artikeln und Whitepapers zu Themen der Informationssicherheit, einschließlich ausführlicher Analysen zum Social Engineering.

\subsection*{eigene Version}

\section{Definition und Angriffsmuster}

\section{Zugangsarten}

\subsection{Elektronischer Zugang}
\subsection{Physischer Zugang}
\subsection{Soziale Medien}
\section{Angriffsvektoren}

\subsection{Phishing in verschiedenen Variationen}
\subsection{Elizitieren}
\subsection{Pretexten}
\subsection{Dumpster diving}
\subsection{Watering Hole}
\subsection{Ködern}
\subsection{Honigtopf}
\subsection{Tailgating/Piggybacking}
\subsection{Business Email Compromise}

\section{Psychologische Prinzipien hinter Social Engineering}

\subsection{stereotypes Verhalten}
\subsection{Reziprozität}
\subsection{Verpflichtung und Konsistenz}
\subsection{Soziale Bewährtheit}
\subsection{Sympathie}
\subsection{Authorität}
\subsection{Knappheit}



\section{Beispiel eines erfolgreichen Social Engineering Angriffs}
